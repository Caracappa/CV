\begin{rSubsection}{Lawrence Berkeley National Laboratory}{Jan.\ 2021 -- present}{Division Director}{Berkeley, CA}
\item Managing the \href{https://www.cyclotronroad.org/}{Cyclotron Road} division to translate hard science into positive societal impact
\item Division hosts a 2-year fellowship program welcoming $\sim$10 new hard tech innovators per year to turn their technology concept into a product ($\sim$\$6M/year)
\item Manage expectations and deliverables across Department of Energy, LBNL, and Activate.org (our partner organization) leadership
\end{rSubsection}

\begin{rSubsection}{University of California, Berkeley}{Jan.\ 2014 - present}{Associate Professor of Nuclear Engineering}{Berkeley, CA}
\item Developed numerical methods for neutral particle transport with an emphasis on supercomputing and advanced architectures; specialization in deterministic, Monte Carlo, and hybrid methods
\item Managed optimization, thermal fluids, and cryptography and anomaly detection projects
\item Applications in reactor design, shielding, and nuclear security and nonproliferation
\item Frequent invited speaker on innovation in the nuclear energy sector
\item Won $>$\$2.5M as principal investigator (PI) and $>$\$26M awarded as co-PI
\item Published 25 journal publications, 44 refereed conference proceedings, 3 technical reports, 2 book chapters, 5 open source pieces of software, and 2 policy pieces
\item Graduated 6 PhD and 6 MS students; research adviser for 1 assistant project scientist, 1 postdoctoral scholar, 1 visiting scholar, and 14 undergraduate students
%\item \href{http://citris-uc.org/decse-mission/}{Design Emphasis in Computational Science and Engineering} Affiliated Faculty member
%\item \href{http://ast.coe.berkeley.edu/}{Applied Science \& Technology} Faculty member
\end{rSubsection}

%------------------------------------------------

\begin{rSubsection}{Advanced Research Projects Agency -- Energy}{Oct.\ 2017 -- Nov.\ 2020}{Program Director}{Washington, DC}
\item Created programs supporting development of enabling technologies for advanced nuclear fission reactors, \$95M across 28 teams: 
\href{https://arpa-e.energy.gov/?q=arpa-e-programs/meitner}{MEITNER}, the
Nuclear
\href{https://arpa-e.energy.gov/?q=news-item/arpa-e-announces-12-million-five-projects-nuclear-materials-science}{OPEN+
cohort},
\href{https://arpa-e.energy.gov/?q=news-item/arpa-e-innovating-through-unconventional-ideas}{LISE}, and  \href{https://arpa-e.energy.gov/technologies/programs/gemina}{GEMINA}
\item Managed \href{https://arpa-e.energy.gov/?q=arpa-e-programs/terra}{TERRA} and \href{https://arpa-e.energy.gov/?q=arpa-e-programs/roots}{ROOTS} Programs, supporting research for sensing and data analytics for above- and below-ground plant outcomes ($\sim$\$90M across 18 teams)
\item Managed \href{https://arpa-e.energy.gov/?q=arpa-e-programs/focus}{FOCUS} Program, supporting research for solar technologies that combine photovoltaic and concentrated solar power technologies ($\sim$\$12M across 4 teams)
\end{rSubsection}

%------------------------------------------------

\begin{rSubsection}{Bettis Laboratory}{Mar.\ 2012 - Aug.\ 2014}{Senior Engineer in the Shield Design and Development group}{West Mifflin, PA}
\item Implemented the Forward-Weighted Consistent Adjoint Driven Importance Sampling (FW-CADIS) method and developed new resonance factor for variance reduction in Monte Carlo
\item Qualified methods and software for use in shield design to dramatically reduce time and improve accuracy in design calculations
\end{rSubsection}

%------------------------------------------------

\begin{rSubsection}{University of Wisconsin--Madison}{Sept.\ 2006 - Nov.\ 2011}{Research Assistant / Rickover Fellow}{Madison, WI}
\item Dissertation: ``Acceleration Methods for Massively Parallel Deterministic Transport" where I added 3 new methods to Denovo, software from Oak Ridge National Laboratory, that are still used%: added parallelization in the energy domain, an advanced eigenvalue solver, and a new multigrid in energy preconditioner to Denovo, developed at Oak Ridge National Lab
%\item Investigated effects of uncertainty in gamma sources on Monte Carlo transport
\item Developed two Monte Carlo source sampling methods for arbitrarily shaped plasma sources%; the sources are generated directly from plasma physics data
%\item Wrote an efficient sampling method for secondary photon sources in the Monte Carlo CAD software library, Direct Accelerated Geometry Monte Carlo (DAGMC)
\end{rSubsection}

%------------------------------------------------

%\begin{rSubsection}{Knolls Atomic Power Laboratory}{Summer 2009}{Rickover Fellow Practicum}{Schenectady, NY}
%\item Helped couple the Monte Carlo code MC21 to the deterministic code Jaguar via the interface code UNIK for the purpose of variance reduction
%\item Added weight window capability to MC21 
%\item Investigated and recommended methods for creating weight window values 
%\end{rSubsection}

%------------------------------------------------

%\begin{rSubsection}{Forschungszentrum Karlsruhe (KIT)}{May 2008 - Dec.\ 2008}{Visiting Researcher}{Karlsruhe, Germany}
%\item Learned about the Rigorous 2 Step method for Monte Carlo geometry conversion while working in the Reactor Safety group
%\item Helped group incorporate DAGMC library into MCNP workflow
%\end{rSubsection}

%------------------------------------------------

%\begin{rSubsection}{Oak Ridge National Laboratory}{Summers 2005 \& 2006}{Summer Intern}{Oak Ridge, TN}
%\item Investigated use of the 3-D deterministic transport code TORT for radiation treatment planning (RTP) when using proper cross sections
%%\item Demonstrated that RTP is possible with TORT, but for limited cases
%\item Learned about electron transport and created electron cross sections with CEPXS
%\end{rSubsection}

%------------------------------------------------

\begin{rSubsection}{Penn State Breazeale Reactor}{Aug.\ 2003 - Apr.\ 2006}{Reactor Operator}{University Park, PA}
\item NRC licensed Reactor Operator for TRIGA Mark III reactor
\item Analyzed core burn-up anomaly; calibrated gamma irradiation facilities
\end{rSubsection}
