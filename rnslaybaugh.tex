%%%%%%%%%%%%%%%%%%%%%%%%%%%%%%%%%%%%%%%%%
% Medium Length Professional CV
% LaTeX Template
% Version 2.0 (8/5/13)
%
% This template has been downloaded from:
% http://www.LaTeXTemplates.com
%
% Original author:
% Trey Hunner (http://www.treyhunner.com/)
%
% Important note:
% This template requires the resume.cls file to be in the same directory as the
% .tex file. The resume.cls file provides the resume style used for structuring the
% document.
%
%%%%%%%%%%%%%%%%%%%%%%%%%%%%%%%%%%%%%%%%%

%----------------------------------------------------------------------------------------
%	PACKAGES AND OTHER DOCUMENT CONFIGURATIONS
%----------------------------------------------------------------------------------------

\documentclass{resume2} % Use the custom resume.cls style

\usepackage[left=0.75in,top=0.6in,right=0.75in,bottom=0.6in]{geometry} % Document margins

\name{Rachel N. Slaybaugh} % Your name
\address{Department of Nuclear Engineering \\ University of California, Berkeley} % Your address
\address{4173 Etcheverry Hall MC 1730 \\ Berkeley, CA 94720} % Your secondary addess (optional)
\address{(570)~$\cdot$~850~$\cdot$~3385 \\ slaybaugh@berkeley.edu} % Your phone number and email

\begin{document}

%----------------------------------------------------------------------------------------
%	EDUCATION SECTION
%----------------------------------------------------------------------------------------

\begin{rSection}{Education}

\begin{tabbing}
Ph.D. \hspace*{2 em}\= \textbf{University of Wisconsin--Madison} \hspace*{5em} \= \hspace*{15em} \= 2011 \\
      \> Nuclear Engineering and Engineering Physics, with a certificate in \\ \> Energy Analysis and Policy \> GPA: 3.971\\
%
M.S. \hspace*{2 em}\> \textbf{University of Wisconsin--Madison} \> \> 2008 \\
      \> Nuclear Engineering and Engineering Physics \> GPA: 3.958\\
%
B.S. \hspace*{2 em}\> \textbf{Pennsylvania State University} \> \> 2006 \\
      \> Nuclear Engineering \> GPA: 3.67\\
\end{tabbing}

\end{rSection}

%----------------------------------------------------------------------------------------
%	RESEARCH EXPERIENCE SECTION
%----------------------------------------------------------------------------------------

\begin{rSection}{Research Experience}

\begin{rSubsection}{University of California, Berkeley}{Jan.\ 2014 - Present}{Assistant Professor of Nuclear Engineering}{Berkeley, CA}
\item Researching transport methods for applications in safeguards and security, shield design, and reactor physics
\end{rSubsection}

%------------------------------------------------

\begin{rSubsection}{Bettis Laboratory}{Mar.\ 2012 - Aug.\ 2014}{Senior Engineer in the Shield Design and Development group}{West Mifflin, PA}
\item Implemented the Forward-Weighted Consistent Adjoint Driven Importance Sampling (FW-CADIS) method for variance reduction in Monte Carlo; accredited method for use in shield design
\item Developed new Resonance Factor variance reduction method for streaming through materials with space and energy self-shielding
\item Built two software tools in support of using FW-CADIS for shield design
\item Scientific Software Development Committee: leader in developing internal website for sharing software carpentry tools and resources
\end{rSubsection}

%------------------------------------------------

\begin{rSubsection}{University of Wisconsin--Madison}{Sept.\ 2006 - Nov.\ 2011}{Research Assistant / Rickover Fellow}{Madison, WI}
\item Researched “Acceleration Methods for Massively Parallel Deterministic Transport”: added parallelization in the energy domain, an advanced eigenvalue solver, and a new multigrid in energy preconditioner to Denovo, developed at Oak Ridge National Lab
\item Investigated effects of uncertainty in gamma sources on Monte Carlo transport
\item Developed two Monte Carlo source sampling methods for arbitrarily shaped plasma sources; the sources are generated directly from plasma physics data
\item Wrote an efficient sampling method for secondary photon sources in the Monte Carlo CAD software library, Direct Accelerated Geometry Monte Carlo (DAGMC)
\end{rSubsection}

%------------------------------------------------

\begin{rSubsection}{Knolls Atomic Power Laboratory}{Summer 2009}{Rickover Fellow Practicum}{Schenectady, NY}
\item Helped couple the Monte Carlo code MC21 to the deterministic code Jaguar via the interface code UNIK for the purpose of variance reduction
\item Added weight window capability to MC21 
\item Investigated and recommended methods for creating weight window values 
\end{rSubsection}

%------------------------------------------------

\begin{rSubsection}{Forschungszentrum Karlsruhe (KIT)}{May 2008 - Dec.\ 2008}{Visiting Researcher}{Karlsruhe, Germany}
\item Learned about the Rigorous 2 Step method for Monte Carlo geometry conversion while working in the Reactor Safety group
\item Helped group incorporate DAGMC library into MCNP workflow
\end{rSubsection}

%------------------------------------------------

\begin{rSubsection}{Oak Ridge National Laboratory}{Summers 2005 \& 2006}{Summer Intern}{Oak Ridge, TN}
\item Investigated use of the 3-D deterministic transport code TORT for radiation treatment planning (RTP) when using proper cross sections
\item Demonstrated that RTP is possible with TORT, but for limited cases
\item Learned about electron transport and created electron cross sections with CEPXS
\end{rSubsection}

%------------------------------------------------

\begin{rSubsection}{Penn State Breazeale Reactor}{Jan.\ 2003 - Apr.\ 2006}{Reactor Operator}{University Park, PA}
\item NRC licensed Reactor Operator for a TRIGA Mark III reactor
\item Analyzed core burn-up anomaly; calibrated gamma irradiation facilities
\end{rSubsection}

\end{rSection}

%----------------------------------------------------------------------------------------
%	TEACHING EXPERIENCE SECTION
%----------------------------------------------------------------------------------------

\begin{rSection}{TEACHING Experience}

\begin{rSubsection}{University of California, Berkeley}{Jan.\ 2014 - Present}{Assistant Professor of Nuclear Engineering}{Berkeley, CA}
\item Taught NE 155, Introduction to Numerical Simulations for Radiation Transport, Spring 2014
\end{rSubsection}

%------------------------------------------------

\begin{rSubsection}{Software Carpentry Scientific Computing Workshop}{Apr. 14-15, 2014}{Instructor}{Berkeley, CA}
\item Hosted by Lawrence Berkeley National Laboratory
\item Introductory material, Version Control, Object Oriented Concepts in Python
\end{rSubsection}


%------------------------------------------------

\begin{rSubsection}{Bettis Laboratory}{Mar.\ 2012 - Aug.\ 2014}{Senior Engineer in the Shield Design and Development group}{West Mifflin, PA}
\item Qualified instructor for Bettis Reactor Engineering School (BRES), an internal school for new DOE-Naval Reactors employees
\item Co-taught BRES Shielding course Fall 2012, 2013 and Spring 2013
\item Used internally-written shielding text by R. Amato
\end{rSubsection}

%------------------------------------------------

\begin{rSubsection}{University of Pittsburgh}{Fall 2012, Spring 2013}{Adjunct Professor}{Pittsburgh, PA}
\item Co-taught Introduction to Nuclear Engineering (ENGR 1700), which covers theory / basic nuclear engineering, basics of nuclear power reactors, and nuclear power reactor operations
\item Co-taught \textit{new} course Nuclear Chemistry and Radiochemistry (ENGR 2112, Spring): responsible for nuclear astrophysics and migration of radionuclides through the environment
\end{rSubsection}

%------------------------------------------------

\begin{rSubsection}{Virtual Science Challenge}{Apr.\ 2012-Mar.\ 2013}{Mentor}{Monterey, CA}
\item Mentor for winning U.S.\ team in international nuclear nonproliferation science fair, organized through the Center for Nonproliferation Studies
\item Facilitated online discussions, participated in workshops, and served as an information resource for the students throughout their project
\end{rSubsection}

%------------------------------------------------

\begin{rSubsection}{Penn State Breazeale Reactor}{Jan.\ 2003 - Apr.\ 2006}{Educational Outreach}{University Park, PA}
\item Taught Freshman Seminar and led tours for kindergarten through college students
\item Developed educational tools such as handouts, presentations, and games
\end{rSubsection}

\end{rSection}



%----------------------------------------------------------------------------------------
%	SELECTED PRESENTATIONS
%----------------------------------------------------------------------------------------

\begin{rSection}{Selected Presentations}

\begin{bibsection}
\item R.N.\ Slaybaugh, T.M.\ Evans, P.P.H.\ Wilson, S.C.\ Wilson. ``Radiation Transport: Computational Methods and Real-World Use.'' N.C.\ State Univ. NE Dept. Graduate Colloquium. Raleigh, NC. 8 Nov.\ 2012. (invited)

\item R.N.\ Slaybaugh. ``Acceleration Methods for Massively Parallel Deterministic Transport." KAPL Employment Meeting. Niskayuna, NY. 30 August 2011. (invited)

\item R.\ Slaybaugh, M.\ Arbidze, S. Lamichhane, D. O’Connor. ``An Evaluation of European Union Energy Policies." UW-Madison Center for World Affairs and the Global Economy Seminar. Madison, WI. 11 May 2011.

\item R.N.\ Slaybaugh. ``Krylov Methods and JFNK." UW-Madison Radiation Hydrodynamics Meeting. Madison, WI. 16 Dec 2010.

\item R.N.\ Slaybaugh, T.M.\ Evans, G.G.\ Davidson. ``Parallel Algorithms for Fixed-Source and Eigenvalue Problems." 2010 SIAM Annual Meeting. Pittsburgh, PA. 12-16 July 2010.

\item R.N.\ Slaybaugh. ``Variance Reduction in MC21 using Forward Adjoint Variance Reduction (FAVRE)." Naval Reactors – Shielding Video-conference. Pittsburgh, PA. August 2010.

\item R.N.\ Slaybaugh. ``MC21 – Jaguar Coupling for Variance Reduction." KAPL Physics Forum. Niskayuna, NY. July 2009.

\end{bibsection}

\end{rSection}

%----------------------------------------------------------------------------------------
%	HONORS AND AWARDS
%----------------------------------------------------------------------------------------

\begin{rSection}{Honors And Awards}

\begin{tabular}{ @{} >{\bfseries}l @{\hspace{6ex}} l }
Languages & C++, Fortran 90/95/2003, Python, MATLAB \\
Version Control & git, svn, cvs \\
Test Frameworks & CTest, GoogleTest, nose \\
Tools & Doxygen, \LaTeX, MathCAD, Mathematica, MCNP, the shell, Vim, tcsh, bash, \\&Emacs, Trilinos, LAPACK, MPI, Valgrind \\
Related & PyNE: Python for Nuclear Engineering (http://pyne.github.com/) participant; \\ & co-author of poster at SciPy 2012; co-organizer of workshop for M\&C 2013
\end{tabular}

\end{rSection}

%----------------------------------------------------------------------------------------
%	COMPUTER SKILLS
%----------------------------------------------------------------------------------------

\begin{rSection}{Computer Skills}

\begin{tabular}{ @{} >{\bfseries}l @{\hspace{6ex}} l }
Languages & C++, Fortran 90/95/2003, Python, MATLAB \\
Version Control & git, svn, cvs \\
Test Frameworks & CTest, GoogleTest, nose \\
Tools & Doxygen, \LaTeX, MathCAD, Mathematica, MCNP, the shell, Vim, tcsh, bash, \\&Emacs, Trilinos, LAPACK, MPI, Valgrind
\end{tabular}

\end{rSection}

%----------------------------------------------------------------------------------------
%	EXAMPLE SECTION
%----------------------------------------------------------------------------------------

%\begin{rSection}{Section Name}

%Section content\ldots

%\end{rSection}

%----------------------------------------------------------------------------------------

\end{document}
