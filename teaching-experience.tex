\begin{rSubsection}{University of California, Berkeley}{Jan.\ 2014 - Present}{Assistant Professor of Nuclear Engineering}{Berkeley, CA}
\item Founder \href{http://nuclearinnovationalliance.org/bootcamp}{Nuclear
Innovation Bootcamp}, Su.\ 2016 - 2020 
\item \href{https://github.com/rachelslaybaugh/NE250 }{NE 250}, Nuclear Reactor Theory (graduate-level): Fa.\ 2015, 2017
\item \href{https://github.com/rachelslaybaugh/NE255}{NE 255}, Numerical Simulations for Radiation Transport (graduate-level): Fa.\ 2016
\item \href{https://github.com/rachelslaybaugh/NE155}{NE 155}, Introduction to
Numerical Simulations for Radiation Transport (senior-level elective): Sp.\
2014, 2015, 2016, 2017, 2021
\item NE 100, Introduction to Nuclear Energy and Techn (junior-level required):
Fa.\ 2020 
\item \href{http://soesberkeley.weebly.com/}{NE 198}, Faculty sponsor for class
in which Berkeley students do hands-on science experiments with students in
under-served elementary schools in Oakland: Fa.\ 2015 - present
\item \href{https://github.com/rachelslaybaugh/NE24}{NE 24}, Putting the Science in Computational Science (Freshman seminar), Sp.\ 2015, 2016, 2017
\end{rSubsection}

%------------------------------------------------

\begin{rSubsection}{Software Carpentry Scientific Computing Workshops}{}{Instructor}{}
\item Jan.\ 14-15, 2016: git; Berkeley Institute for Data Science
\item July 16, 2015: shell; \'{E}cole Polytechnique F\'{e}d\'{e}rale Lausanne
\item July 1-2, 2015: shell and Python; for underrepresented minority students; UC, Berkeley
\item June 11, 2015: Python; Oak Ridge National Laboratory
\item Jan.\ 5-6, 2015: version control; for women only; University of Colorado, Boulder
\item Apr.\ 14-15, 2014: introductory material, version control, object oriented concepts in Python; for women only; Lawrence Berkeley National Laboratory
\end{rSubsection}


%------------------------------------------------

\begin{rSubsection}{Bettis Laboratory}{Mar.\ 2012 - Aug.\ 2014}{Instructor}{West Mifflin, PA}
\item Qualified instructor for Bettis Reactor Engineering School (BRES), an internal school for new DOE-Naval Reactors employees
\item Co-taught BRES Shielding course Fa.\ 2012, 2013, and Sp.\ 2013
%\item Used internally-written shielding text by R.\ Amato
\end{rSubsection}

%------------------------------------------------
%\clearpage
\begin{rSubsection}{University of Pittsburgh}{Fa.\ 2012, Sp.\ 2013}{Adjunct Professor}{Pittsburgh, PA}
\item Co-taught Introduction to Nuclear Engineering (ENGR 1700), %which covers theory / basic nuclear engineering, basics of nuclear power reactors, and nuclear power reactor operations 
Fa.\ 2012
\item Co-taught \textit{new} course Nuclear Chemistry and Radiochemistry (ENGR
2112), %responsible for nuclear astrophysics and migration of radionuclides through the environment 
Sp.\ 2013
\end{rSubsection}

%------------------------------------------------

%\begin{rSubsection}{Virtual Science Challenge}{Apr.\ 2012-Mar.\ 2013}{Mentor}{Monterey, CA}
%\item Mentor for winning U.S.\ team in international nuclear nonproliferation science fair, organized through the Center for Nonproliferation Studies
%\item Facilitated online discussions, participated in workshops, and served as an information resource for the students throughout their project
%\end{rSubsection}

%------------------------------------------------
%
%\begin{rSubsection}{Penn State Breazeale Reactor}{Jan.\ 2003 - Apr.\ 2006}{Educational Outreach}{University Park, PA}
%\item Taught Freshman Seminar and led tours for kindergarten through college students
%\item Developed educational tools such as handouts, presentations, and games
%\end{rSubsection}
