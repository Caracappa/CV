\begin{rSubsection}{University of California, Berkeley}{Jan.\ 2014 - Present}{Assistant Professor of Nuclear Engineering}{Berkeley, CA}
\item Researching numerical methods for neutral particle transport with an emphasis on supercomputing and advanced architectures as well as data science
\item Specialization in deterministic, Monte Carlo, and hybrid methods
\item Applications in reactor design, shielding, and nuclear security and nonproliferation
\item \href{http://citris-uc.org/decse-mission/}{Design Emphasis  in Computational Science and Engineering} Affiliated Faculty member
\item \href{http://ast.coe.berkeley.edu/}{Applied Science \& Technology} Faculty member
\end{rSubsection}

%------------------------------------------------

\begin{rSubsection}{Advanced Research Projects Agency -- Energy}{Oct.\ 2017 -- present}{Program Director}{Washington, DC}
\item Program creation and management
\item Director for \href{https://arpa-e.energy.gov/?q=arpa-e-programs/meitner}{MEITNER}, the Nuclear \href{https://arpa-e.energy.gov/?q=news-item/arpa-e-announces-12-million-five-projects-nuclear-materials-science}{OPEN+ cohort}, LISE, and  \href{https://arpa-e.energy.gov/?q=workshop/optimal-operations-advanced-nuclear-reactors}{GEMINA} Programs, supporting research for enabling technologies for advanced nuclear fission reactors
\item Director for \href{https://arpa-e.energy.gov/?q=arpa-e-programs/terra}{TERRA} and \href{https://arpa-e.energy.gov/?q=arpa-e-programs/roots}{ROOTS} Programs, supporting research for sensing and data analytics for above- and below-ground plant outcomes
\item Director for \href{https://arpa-e.energy.gov/?q=arpa-e-programs/focus}{FOCUS} Program, supporting research for solar technologies that combine photovoltaic and concentrated solar power technologies
\end{rSubsection}


%------------------------------------------------

\begin{rSubsection}{Bettis Laboratory}{Mar.\ 2012 - Aug.\ 2014}{Senior Engineer in the Shield Design and Development group}{West Mifflin, PA}
\item Implemented the Forward-Weighted Consistent Adjoint Driven Importance Sampling (FW-CADIS) method for variance reduction in Monte Carlo; accredited method for use in shield design
\item Developed new Resonance Factor variance reduction method for streaming through materials with space and energy self-shielding
\item Built two software tools in support of using FW-CADIS for shield design
\item Scientific Software Development Committee: leader in developing internal website for sharing software carpentry tools and resources
\end{rSubsection}

%------------------------------------------------
\clearpage
\begin{rSubsection}{University of Wisconsin--Madison}{Sept.\ 2006 - Nov.\ 2011}{Research Assistant / Rickover Fellow}{Madison, WI}
\item Researched ``Acceleration Methods for Massively Parallel Deterministic Transport": added parallelization in the energy domain, an advanced eigenvalue solver, and a new multigrid in energy preconditioner to Denovo, developed at Oak Ridge National Lab
%\item Investigated effects of uncertainty in gamma sources on Monte Carlo transport
\item Developed two Monte Carlo source sampling methods for arbitrarily shaped plasma sources; the sources are generated directly from plasma physics data
%\item Wrote an efficient sampling method for secondary photon sources in the Monte Carlo CAD software library, Direct Accelerated Geometry Monte Carlo (DAGMC)
\end{rSubsection}

%------------------------------------------------

%\begin{rSubsection}{Knolls Atomic Power Laboratory}{Summer 2009}{Rickover Fellow Practicum}{Schenectady, NY}
%\item Helped couple the Monte Carlo code MC21 to the deterministic code Jaguar via the interface code UNIK for the purpose of variance reduction
%\item Added weight window capability to MC21 
%\item Investigated and recommended methods for creating weight window values 
%\end{rSubsection}

%------------------------------------------------

\begin{rSubsection}{Forschungszentrum Karlsruhe (KIT)}{May 2008 - Dec.\ 2008}{Visiting Researcher}{Karlsruhe, Germany}
\item Learned about the Rigorous 2 Step method for Monte Carlo geometry conversion while working in the Reactor Safety group
\item Helped group incorporate the Direct Accelerated Geometry Monte Carlo (DAGMC) library into MCNP workflow
\end{rSubsection}

%------------------------------------------------

%\begin{rSubsection}{Oak Ridge National Laboratory}{Summers 2005 \& 2006}{Summer Intern}{Oak Ridge, TN}
%\item Investigated use of the 3-D deterministic transport code TORT for radiation treatment planning (RTP) when using proper cross sections
%%\item Demonstrated that RTP is possible with TORT, but for limited cases
%\item Learned about electron transport and created electron cross sections with CEPXS
%\end{rSubsection}

%------------------------------------------------

\begin{rSubsection}{Penn State Breazeale Reactor}{Aug.\ 2003 - Apr.\ 2006}{Reactor Operator}{University Park, PA}
\item NRC licensed Reactor Operator for TRIGA Mark III reactor
\item Analyzed core burn-up anomaly; calibrated gamma irradiation facilities
\end{rSubsection}
